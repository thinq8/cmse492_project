%%%%%%%%%%%%%%%%%%%%%%%%%%%%%%%%%%%%%%%%%%%%%%%%%%%%%%%%%%%%%%%%%%%%%%%%%%%%%%%
% CMSE 492 Final Project Report Template
% Using RevTeX 4.2 for professional scientific document formatting
%%%%%%%%%%%%%%%%%%%%%%%%%%%%%%%%%%%%%%%%%%%%%%%%%%%%%%%%%%%%%%%%%%%%%%%%%%%%%%%

\documentclass[aps,prl,preprint,groupedaddress]{revtex4-2}

% Essential packages
\usepackage{graphicx}       % For including figures
\usepackage{dcolumn}        % Align table columns on decimal point
\usepackage{bm}             % Bold math symbols
\usepackage{hyperref}       % Hyperlinks
\usepackage{amsmath}        % Advanced math features
\usepackage{amssymb}        % Math symbols
\usepackage{booktabs}       % Professional-looking tables
\usepackage{float}          % Better float placement
\usepackage{caption}        % Caption customization
\usepackage{subcaption}     % Subfigures
\usepackage{listings}       % Code listings (optional)
\usepackage{xcolor}         % Colors

% Hyperlink setup
\hypersetup{
    colorlinks=true,
    linkcolor=blue,
    filecolor=magenta,      
    urlcolor=cyan,
    citecolor=blue,
}

% Code listing setup (optional - uncomment if needed)
% \lstset{
%     basicstyle=\ttfamily\small,
%     breaklines=true,
%     frame=single,
%     language=Python,
%     showstringspaces=false,
%     commentstyle=\color{green!50!black},
%     keywordstyle=\color{blue},
%     stringstyle=\color{red}
% }

%%%%%%%%%%%%%%%%%%%%%%%%%%%%%%%%%%%%%%%%%%%%%%%%%%%%%%%%%%%%%%%%%%%%%%%%%%%%%%%
% DOCUMENT INFORMATION - FILL IN YOUR DETAILS
%%%%%%%%%%%%%%%%%%%%%%%%%%%%%%%%%%%%%%%%%%%%%%%%%%%%%%%%%%%%%%%%%%%%%%%%%%%%%%%

\begin{document}

\title{[Your Project Title Here]}

\author{[Your Name]}
\email{[your.email@msu.edu]}
\affiliation{Department of Computational Mathematics, Science and Engineering\\
Michigan State University, East Lansing, MI 48824}

\date{\today}

\begin{abstract}
Write a concise abstract (150-250 words) summarizing your project. Include: (1) the problem you addressed, (2) the machine learning approach you used, (3) the key findings or results, and (4) the implications of your work. The abstract should be understandable without reading the full report.
\end{abstract}

\maketitle

%%%%%%%%%%%%%%%%%%%%%%%%%%%%%%%%%%%%%%%%%%%%%%%%%%%%%%%%%%%%%%%%%%%%%%%%%%%%%%%
\section{Background and Motivation}
\label{sec:background}
%%%%%%%%%%%%%%%%%%%%%%%%%%%%%%%%%%%%%%%%%%%%%%%%%%%%%%%%%%%%%%%%%%%%%%%%%%%%%%%

Describe the problem/question you are attempting to answer. This section must answer the following questions:

\begin{itemize}
    \item Why is this problem/question important?
    \item Who cares about this problem/question being solved/answered?
    \item What are the consequences of solving this problem/answering this question?
    \item What has been done so far to address this problem/question?
    \item State very clearly what the desired outcome is. How can Machine Learning (ML) help achieve your goal and/or solve your problem?
\end{itemize}

[Write your background and motivation here. Use multiple paragraphs to organize your thoughts. Include citations where appropriate using \textbackslash cite\{key\}.]

%%%%%%%%%%%%%%%%%%%%%%%%%%%%%%%%%%%%%%%%%%%%%%%%%%%%%%%%%%%%%%%%%%%%%%%%%%%%%%%
\section{Data Description}
\label{sec:data}
%%%%%%%%%%%%%%%%%%%%%%%%%%%%%%%%%%%%%%%%%%%%%%%%%%%%%%%%%%%%%%%%%%%%%%%%%%%%%%%

Describe your data and any issues there might be. This section should have clear answers to all these questions:

\subsection{Data Origins}
This does not mean ``I got the data from Kaggle.'' Instead, you should read the description and metadata of the dataset and report that. For example: ``The MNIST dataset consists of 60,000 images of handwritten digits written by 500 high school students in Bethesda. The dataset was originally assembled by the US Census Bureau in the 1990s.''

[Describe the origin and context of your dataset here.]

\subsection{Dataset Characteristics}
\begin{itemize}
    \item Number of samples (rows): [X]
    \item Number of features (columns): [Y]
    \item Data types: [Numerical, categorical, time series, geographical, etc.]
    \item Target variable: [Describe your target variable]
\end{itemize}

\subsection{Data Quality Analysis}

\subsubsection{Missing Values}
Are there missing values? What do you think is the missingness mechanism? Pattern? How did you arrive at this conclusion?

[Your analysis here.]

\subsubsection{Class Balance}
Is the dataset balanced? What technique are you going to use to balance the dataset if needed?

[Your analysis here.]

\subsubsection{Statistical Summary}
Show some statistics of the data: correlations, univariate and bivariate distributions, ranges of the data, outliers.

[Include figures and tables here. Example:]
% \begin{figure}[H]
%     \centering
%     \includegraphics[width=0.8\linewidth]{figures/correlation_matrix.png}
%     \caption{Correlation matrix of features.}
%     \label{fig:correlation}
% \end{figure}

%%%%%%%%%%%%%%%%%%%%%%%%%%%%%%%%%%%%%%%%%%%%%%%%%%%%%%%%%%%%%%%%%%%%%%%%%%%%%%%
\section{Preprocessing}
\label{sec:preprocessing}
%%%%%%%%%%%%%%%%%%%%%%%%%%%%%%%%%%%%%%%%%%%%%%%%%%%%%%%%%%%%%%%%%%%%%%%%%%%%%%%

Describe the preprocessing steps and why you are doing these steps.

\subsection{Data Splitting}
How are you going to split the data and why did you choose it? Stratified splitting, random splitting, time series splitting? Recall that the splitting should happen before you do any EDA.

[Your explanation here.]

\subsection{Feature Engineering}
Describe any feature engineering techniques. For example: ``We used K-means clustering to create 5 clusters of the CA districts,'' or ``We created polynomials up to degree 10 for all the features.''

[Your feature engineering approach here.]

\subsection{Scaling, Transformation, and Encoding}
Describe any scaling, transformation, encoding, or imputation techniques used.

[Your preprocessing pipeline here.]

%%%%%%%%%%%%%%%%%%%%%%%%%%%%%%%%%%%%%%%%%%%%%%%%%%%%%%%%%%%%%%%%%%%%%%%%%%%%%%%
\section{Machine Learning Task and Objective}
\label{sec:ml_task}
%%%%%%%%%%%%%%%%%%%%%%%%%%%%%%%%%%%%%%%%%%%%%%%%%%%%%%%%%%%%%%%%%%%%%%%%%%%%%%%

This section focuses on the machine learning aspect of the project.

\subsection{Why Machine Learning?}
Describe why we need ML and how humans or current methods fail at this task.

[Your justification here.]

\subsection{Task Type}
What type of ML task is this?

\begin{itemize}
    \item \textbf{Supervised Learning:}
    \begin{itemize}
        \item Regression: [Interpolation/Extrapolation]
        \item Classification: [Binary/Multiclass/Multi-label/Multi-output]
    \end{itemize}
    \item \textbf{Unsupervised Learning:}
    \begin{itemize}
        \item Dimensionality Reduction
        \item Clustering
    \end{itemize}
    \item \textbf{Reinforcement Learning:}
    \begin{itemize}
        \item [Value-based/Policy-based/Actor-critic/Policy-learning]
    \end{itemize}
\end{itemize}

[Specify and explain your task type.]

%%%%%%%%%%%%%%%%%%%%%%%%%%%%%%%%%%%%%%%%%%%%%%%%%%%%%%%%%%%%%%%%%%%%%%%%%%%%%%%
\section{Models}
\label{sec:models}
%%%%%%%%%%%%%%%%%%%%%%%%%%%%%%%%%%%%%%%%%%%%%%%%%%%%%%%%%%%%%%%%%%%%%%%%%%%%%%%

Describe the machine learning models you will compare. You need at least three models in increasing order of complexity.

\subsection{Model Selection}
Describe the models you are going to use and how they will be evaluated. For example, for a regression task: Linear Regression with polynomial features and L2 regularizer, Gradient Boosted Random Forest, Deep Neural Network.

\subsubsection{Model 1: [Simple Model Name]}
[Description, rationale, and key characteristics]

\subsubsection{Model 2: [Intermediate Model Name]}
[Description, rationale, and key characteristics]

\subsubsection{Model 3: [Complex Model Name]}
[Description, rationale, and key characteristics. If using neural networks, describe the architecture and why you chose it.]

\subsection{Regularization and Hyperparameter Tuning}
Describe the regularization and hyperparameter tuning procedures if any.

[Your approach here.]

%%%%%%%%%%%%%%%%%%%%%%%%%%%%%%%%%%%%%%%%%%%%%%%%%%%%%%%%%%%%%%%%%%%%%%%%%%%%%%%
\section{Training Methodology}
\label{sec:training}
%%%%%%%%%%%%%%%%%%%%%%%%%%%%%%%%%%%%%%%%%%%%%%%%%%%%%%%%%%%%%%%%%%%%%%%%%%%%%%%

For each model, describe how training is performed, write down the equation for the loss function, and any technique used to track the learning of your model and avoid over- and under-fitting.

\subsection{Loss Functions}
For each model, specify the loss function. Example:

\textbf{Model 1 (Linear Regression):}
\begin{equation}
\mathcal{L}(\mathbf{w}) = \frac{1}{n}\sum_{i=1}^{n}(y_i - \mathbf{w}^T\mathbf{x}_i)^2 + \lambda||\mathbf{w}||_2^2
\end{equation}

[Write the loss functions for each of your models.]

\subsection{Training Process}
This section should include hyperparameter tuning, cross-validation, bootstrapping, etc. Include plots of learning curves or other metrics used to track the learning process.

% Example figure for learning curves
% \begin{figure}[H]
%     \centering
%     \includegraphics[width=0.8\linewidth]{figures/learning_curves.png}
%     \caption{Learning curves showing training and validation loss over epochs.}
%     \label{fig:learning_curves}
% \end{figure}

\subsection{Model Summary Table}
This section must contain a table with these columns:

\begin{table}[H]
\centering
\caption{Summary of models, parameters, and training methodology.}
\label{tab:model_summary}
\begin{tabular}{@{}lllll@{}}
\toprule
\textbf{Model} & \textbf{Parameters} & \textbf{Hyperparameters} & \textbf{Loss Function} & \textbf{Regularization} \\ 
\midrule
Model 1 & [e.g., w, b] & [e.g., $\lambda$=0.01] & [MSE] & [L2] \\
Model 2 & [...] & [...] & [...] & [...] \\
Model 3 & [...] & [...] & [...] & [...] \\
\bottomrule
\end{tabular}
\end{table}

%%%%%%%%%%%%%%%%%%%%%%%%%%%%%%%%%%%%%%%%%%%%%%%%%%%%%%%%%%%%%%%%%%%%%%%%%%%%%%%
\section{Metrics}
\label{sec:metrics}
%%%%%%%%%%%%%%%%%%%%%%%%%%%%%%%%%%%%%%%%%%%%%%%%%%%%%%%%%%%%%%%%%%%%%%%%%%%%%%%

Clearly define the metrics you will be using to evaluate the performance. How do you know that your model is doing well? Examples: RMSE, MSE, F1 Score, precision, recall, accuracy, AUC-ROC, etc.

\subsection{Primary Metric}
[Define your primary evaluation metric and explain why you chose it.]

\subsection{Secondary Metrics}
[Define any additional metrics that provide complementary information about model performance.]

\subsection{Metric Definitions}
Provide mathematical definitions of your metrics. Example:

\begin{equation}
\text{RMSE} = \sqrt{\frac{1}{n}\sum_{i=1}^{n}(y_i - \hat{y}_i)^2}
\end{equation}

[Define your metrics mathematically.]

%%%%%%%%%%%%%%%%%%%%%%%%%%%%%%%%%%%%%%%%%%%%%%%%%%%%%%%%%%%%%%%%%%%%%%%%%%%%%%%
\section{Results and Model Comparison}
\label{sec:results}
%%%%%%%%%%%%%%%%%%%%%%%%%%%%%%%%%%%%%%%%%%%%%%%%%%%%%%%%%%%%%%%%%%%%%%%%%%%%%%%

Compare the different algorithms using the metrics defined above. Compare the algorithms on their difficulty in training (time and hardware resources). Explain your choice of best algorithm for the task.

\subsection{Performance Comparison}

\begin{table}[H]
\centering
\caption{Model performance metrics on test set.}
\label{tab:performance}
\begin{tabular}{@{}lcccc@{}}
\toprule
\textbf{Model} & \textbf{Metric 1} & \textbf{Metric 2} & \textbf{Metric 3} & \textbf{Metric 4} \\ 
\midrule
Model 1 & [value] & [value] & [value] & [value] \\
Model 2 & [value] & [value] & [value] & [value] \\
Model 3 & [value] & [value] & [value] & [value] \\
\bottomrule
\end{tabular}
\end{table}

\subsection{Computational Efficiency}

\begin{table}[H]
\centering
\caption{Training and inference time for each model.}
\label{tab:timing}
\begin{tabular}{@{}lccc@{}}
\toprule
\textbf{Model} & \textbf{Training Time} & \textbf{Inference Time} & \textbf{Hardware Used} \\ 
\midrule
Model 1 & [time] & [time] & [e.g., CPU] \\
Model 2 & [time] & [time] & [e.g., CPU] \\
Model 3 & [time] & [time] & [e.g., GPU] \\
\bottomrule
\end{tabular}
\end{table}

\subsection{Analysis and Discussion}
Explain your choice of best algorithm for the task. Explain why some models perform better than others and/or why all the models are not performing well.

[Your analysis here. Include figures comparing model performance if helpful.]

%%%%%%%%%%%%%%%%%%%%%%%%%%%%%%%%%%%%%%%%%%%%%%%%%%%%%%%%%%%%%%%%%%%%%%%%%%%%%%%
\section{Model Interpretation}
\label{sec:interpretation}
%%%%%%%%%%%%%%%%%%%%%%%%%%%%%%%%%%%%%%%%%%%%%%%%%%%%%%%%%%%%%%%%%%%%%%%%%%%%%%%

Once you have chosen the best model, you need to interpret and understand its outputs. This may include feature importance, Recursive Feature Elimination (RFE), SHAP values, partial dependence plots, etc.

\subsection{Feature Importance}
[Discuss which features are most important for your model's predictions.]

% Example figure
% \begin{figure}[H]
%     \centering
%     \includegraphics[width=0.8\linewidth]{figures/feature_importance.png}
%     \caption{Feature importance scores for the best model.}
%     \label{fig:feature_importance}
% \end{figure}

\subsection{Model Behavior Analysis}
[Discuss how your model makes predictions. Include visualizations such as SHAP plots, decision boundaries, or activation maps if applicable.]

%%%%%%%%%%%%%%%%%%%%%%%%%%%%%%%%%%%%%%%%%%%%%%%%%%%%%%%%%%%%%%%%%%%%%%%%%%%%%%%
\section{Conclusion}
\label{sec:conclusion}
%%%%%%%%%%%%%%%%%%%%%%%%%%%%%%%%%%%%%%%%%%%%%%%%%%%%%%%%%%%%%%%%%%%%%%%%%%%%%%%

Summarize what you have done. Which is the best algorithm for the task and why? Did your algorithm achieve the desired score?

\subsection{Summary of Findings}
[Summarize your main findings and the best model.]

\subsection{Limitations and Future Work}
In addition, describe what went wrong and how you think you could solve the issues in the future.

\subsection{Final Remarks}
[Concluding thoughts on the project.]

%%%%%%%%%%%%%%%%%%%%%%%%%%%%%%%%%%%%%%%%%%%%%%%%%%%%%%%%%%%%%%%%%%%%%%%%%%%%%%%
% ACKNOWLEDGMENTS (Optional)
%%%%%%%%%%%%%%%%%%%%%%%%%%%%%%%%%%%%%%%%%%%%%%%%%%%%%%%%%%%%%%%%%%%%%%%%%%%%%%%
\begin{acknowledgments}
I would like to thank [names] for their help and support. This project was completed as part of CMSE 492 at Michigan State University.
\end{acknowledgments}

%%%%%%%%%%%%%%%%%%%%%%%%%%%%%%%%%%%%%%%%%%%%%%%%%%%%%%%%%%%%%%%%%%%%%%%%%%%%%%%
% REFERENCES
%%%%%%%%%%%%%%%%%%%%%%%%%%%%%%%%%%%%%%%%%%%%%%%%%%%%%%%%%%%%%%%%%%%%%%%%%%%%%%%
% You can use BibTeX for references. Create a .bib file and uncomment below:
% \bibliography{references}

% Or manually add references:
\begin{thebibliography}{99}

\bibitem{example1}
Author Name,
``Title of Paper,''
\textit{Journal Name} \textbf{Volume}, Page (Year).

\bibitem{example2}
Author Name,
``Title of Book,''
Publisher (Year).

% Add your references here

\end{thebibliography}

%%%%%%%%%%%%%%%%%%%%%%%%%%%%%%%%%%%%%%%%%%%%%%%%%%%%%%%%%%%%%%%%%%%%%%%%%%%%%%%
% APPENDIX (Optional)
%%%%%%%%%%%%%%%%%%%%%%%%%%%%%%%%%%%%%%%%%%%%%%%%%%%%%%%%%%%%%%%%%%%%%%%%%%%%%%%
\appendix

\section{Additional Figures and Tables}
\label{app:additional}

[Include any additional supporting material here.]

\section{Code Availability}
\label{app:code}

The complete code for this project is available at: \url{https://github.com/yourusername/your-repo}

\end{document}
